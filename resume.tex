%%%%%%%%%%%%%%%%%
% This is an sample CV template created using altacv.cls
% (v1.1.2, 1 February 2017) written by LianTze Lim (liantze@gmail.com). Now compiles with pdfLaTeX, XeLaTeX and LuaLaTeX.
%
%% It may be distributed and/or modified under the
%% conditions of the LaTeX Project Public License, either version 1.3
%% of this license or (at your option) any later version.
%% The latest version of this license is in
%%    http://www.latex-project.org/lppl.txt
%% and version 1.3 or later is part of all distributions of LaTeX
%% version 2003/12/01 or later.
%%%%%%%%%%%%%%%%

%% If you need to pass whatever options to xcolor
\PassOptionsToPackage{dvipsnames}{xcolor}

%% If you are using \orcid or academicons
%% icons, make sure you have the academicons
%% option here, and compile with XeLaTeX
%% or LuaLaTeX.
% \documentclass[10pt,a4paper,academicons]{altacv}

%% Use the "normalphoto" option if you want a normal photo instead of cropped to a circle
% \documentclass[10pt,a4paper,normalphoto]{altacv}

\documentclass[10pt,letter]{altacv}

%% AltaCV uses the fontawesome and academicon fonts
%% and packages.
%% See texdoc.net/pkg/fontawecome and http://texdoc.net/pkg/academicons for full list of symbols.
%%
%% Compile with LuaLaTeX for best results. If you
%% want to use XeLaTeX, you may need to install
%% Academicons.ttf in your operating system's font
%% folder.


% Change the page layout if you need to
\geometry{left=1cm,right=9cm,marginparwidth=7.25cm,marginparsep=0.75cm,top=0.5cm,bottom=1cm}

% Change the font if you want to.

% If using pdflatex:
\usepackage[utf8]{inputenc}
\usepackage[T1]{fontenc}
\usepackage[default]{lato}
\usepackage[none]{hyphenat}
\usepackage[document]{ragged2e}

% If using xelatex or lualatex:
% \setmainfont{Lato}

% Change the colours if you want to
\definecolor{AccentGreen}{HTML}{339966}
\definecolor{AcePurple}{HTML}{800080}
\definecolor{DarkerPurple}{HTML}{5b005b}
\definecolor{SlateGrey}{HTML}{2E2E2E}
\definecolor{LightGrey}{HTML}{666666}
\definecolor{SomeBlue}{HTML}{4286f4}
\colorlet{heading}{AcePurple}%Sepia}
\colorlet{accent}{DarkerPurple}%AccentGreen}
\colorlet{emphasis}{SlateGrey}
\colorlet{body}{LightGrey}

% Change the bullets for itemize and rating marker
% for \cvskill if you want to
\renewcommand{\itemmarker}{{\small\textbullet}}
\renewcommand{\ratingmarker}{\faCircle}

\begin{document}
\name{Galen Guyer}
% \tagline{Software Engineer Seeking a Spring or Summer 2022 Co-op}
\personalinfo{%
  % Not all of these are required!
  % You can add your own with \printinfo{symbol}{detail}
  \email{galen@galenguyer.com}\\\smallskip
  \phone{(360) 797-9404}\\\smallskip
  \github{galenguyer}\\\smallskip
  \homepage{galenguyer.com}\\\smallskip
  \linkedin{in/galenguyer}\\\smallskip
  %% You MUST add the academicons option to \documentclass, then compile with LuaLaTeX or XeLaTeX, if you want to use \orcid or other academicons commands.
%   \orcid{orcid.org/0000-0000-0000-0000}
}

%% Make the header extend all the way to the right, if you want.
\begin{fullwidth}
\marginpar{\makesidebarheader\vspace{2pt plus 1pt minus 1pt}

\cvsection{Skills}
\cvsubsection{Languages}
\cvtag{Rust}
\cvtag{Go}
\cvtag{React}
\cvtag{Python}
\cvtag{HTML/CSS}
\cvtag{JavaScript}
\cvtag{C\#}
\cvtag{PowerShell}
% \cvtag{Java}

\smallskip
\smallskip

\cvsubsection{Tools}
\cvtag{NGINX}
\cvtag{Linux}
\cvtag{Git}
\cvtag{Docker}
\cvtag{Docker Compose}
\cvtag{Traefik}
\cvtag{Azure}
\cvtag{Bash}
\cvtag{PostgreSQL}
\cvtag{MongoDB}
\cvtag{Flask}
\cvtag{Ansible}
\cvtag{Terraform}
\cvtag{Kubernetes}

%% Yeah I didn't spend too much time making all the
%% spacing consistent... sorry. Use \smallskip, \medskip,
%% \bigskip, \vpsace etc to make ajustments.
\medskip

\cvsection{Education}
\cvevent{Rochester Institute of Technology}{Applied Sciences B.S.}{May 2024}{}

\medskip

\cvsection{Activities}

\cvevent{Computer Science House}{System Administrator}{April 2020 - Present}{}
Maintain House Services such as the student-run server room and all services within it, including an OpenShift cluster, a Proxmox cluster, and FreeIPA user management

\divider

\cvevent{Amateur Radio Operator}{Amateur Expert}{September 2022 - Present}{}
Licensed Amateur Radio Operator with Amateur Expert class license and Volunteer Examiner

\divider

\cvevent{Society of Software Engineers}{Technology Head}{August 2021 - December 2021}{}
Responsible for maintenance and development of the organization's website, server, and other systems

\divider

\cvevent{Accessibility Learning Labs}{Software Engineer}{May 2020 - August 2020}{}
Worked in a federally funded research program to develop React apps to demonstrate the importance of accessible software to students and developers

\divider

\cvevent{Boy Scouts}{Eagle Scout}{September 2017}{}
Achieved Boy Scouts' highest rank after earning 26 merit badges and designing, coordinating, and leading a 100+ hour service project

% \cvevent{Society of Software Engineers}{Projects Head}{August 2022 - December 2022}{}
% Responsible for co-ordinating and supporting projects made by members of the organization, including the organization's website re-write

% \cvevent{Teaching Assistant}{Social Network Analytics and Data Visualization}{August 2021 - December 2021}{}
% Assisted with designing coursework and demonstrations as well as grading completed work

%\cvevent{Society of Software Engineers}{Public Relations}{January 2020 - May 2020, January 2021 - Current}{}
%Coordinated company visits and tech talks throughout the semester

%\cvevent{WITR Radio}{Internal Developer}{October 2019 - Present}{}
%Responsible for the upkeep and development of internal and public facing services such as the public website

}
    \vspace*{-1\baselineskip}
\makecvheader
\end{fullwidth}
%% Provide the file name containing the sidebar contents as an optional parameter to \cvsection.
%% You can always just use \marginpar{...} if you do
%% not need to align the top of the contents to any
%% \cvsection title in the "main" bar.

\cvsection{Experience}

\cvevent{Bryx}{DevOps Engineer}{May 2022 - Present}{Rochester, NY}
\begin{itemize}
  \item Designed and tested a deduplication system migration from Redis to self-hosted CockroachDB for reduced latency and increased reliability
  \item Built a tool to open the database firewall for managed kubernetes nodes to allow for access from within the cluster without manual intervention
\end{itemize}
\textit{\textbf{Tools:} Kubernetes, Puppet, Rust, Go, Python, CockroachDB}

%\divider

\cvevent{Council Rock}{DevOps Intern}{January 2022 - May 2022}{Rochester, NY}
\begin{itemize}
  \item Moved cloud deployments from virtual machines to Kubernetes within AWS
  \item Diagnosed and resolved issues with long-lasting UDP connections caused by Linux kernel configurations
\end{itemize}
\textit{\textbf{Tools:} Kubernetes, Linux}

% \cvevent{Blackbaud}{DevOps Intern - Raiser's Edge NXT}{May 2021 -- August 2021}{Remote}
% \begin{itemize}
%   \item Worked with the Splunk API to automate creation and updating of alerts from version-controlled files
%   \item Designed sane defaults to be loaded by a script to ease creation and updating of alerts in bulk
% \end{itemize}
% \textit{\textbf{Tools:} PowerShell, Splunk}

%\divider

\cvevent{Microsoft}{Software Engineering Intern - One Customer Voice Team}{June 2019 -- August 2019}{Redmond, WA}
\begin{itemize}
\item Improved item grouping for the internal feedback aggregation tool
\item Exposed all previously hidden top level fields and automatically detected field type via ElasticSearch mappings, greatly increasing both the granularity and flexibility for users
% \item Implemented frontend, backend, and tests with 100\% backend code coverage
\end{itemize}
\textit{\textbf{Tools:} C\#, ASP.NET, ElasticSearch, AngularJS}

\cvsubevent{Software Engineering Intern - Office Security Penetration Testing Team}{June 2018 -- August 2018}{Redmond, WA}
\begin{itemize}
\item Developed a PowerShell script to create a Windows VM, automatically install Office, gather debugging symbols for Office, and send these to a Microsoft \\ Security Risk Detection server to install and start a fuzzing run
% \item Implemented a server to collect, deduplicate, and report new bugs
\item Worked closely with the Microsoft Security Risk Detection team to help improve the job deployment process while the tool was in internal beta
\end{itemize}
\textit{\textbf{Tools:} PowerShell, Microsoft Security Risk Detection, Azure, ASP.NET}

\smallskip

\cvsection{Projects}

%\divider

%\project{RIT COVID-19 Dashboard}{https://github.com/galenguyer/rit-covid-dashboard}
\noghproject{RIT COVID-19 Dashboard}{https://ritcoviddashboard.com}
\begin{itemize}
\item Wrote a Python scraper to check the official dashboard on an interval and update a SQLite database with the numbers from the official dashboard, then expose the saved data with a JSON API
\item Consumed data from the scraper in a React app that provides all the information from the official dashboard as well as changes over a user-selected time period and graphs of all historical data
% \item Used by roughly 10\% of the student body within 24 hours of launch
\end{itemize}
\textit{\textbf{Tools:} Python, SQLite, Flask, React, Docker}

%\divider

\project{HOSTSdotTXT}{https://github.com/hostsdottxt}
\begin{itemize}
  \item Wrote an API and authoritative DNS resolver in Rust to serve DNS records \\ from a PostgreSQL database
  \item Achieved a 99th percentile latency for resolving queries of under 2ms with \\ near-instant propogation of changes by using native PostgreSQL replication
\end{itemize}
\textit{\textbf{Tools:} Rust, PostgreSQL, ReactJS}

% \project{GenericBot}{https://github.com/galenguyer/GenericBot}
% \begin{itemize}
% \item Built a Discord bot to provide moderation tools and fun commands to over 70 servers totaling over 30,000 users
% \item Used C\# to connect to Discord's API and store all user data in an encrypted, self-hosted MongoDB instance
% % \item Exposed user-stored quotes via a webpage ASP.NET Razor Pages for server-side rendering and JavaScript for fast, responsive search
% \end{itemize}
% \textit{\textbf{Tools:} C\#, MongoDB, ASP.NET, Razor Pages, HTML/CSS/JS, Azure Build Pipelines}

% \noghproject{Global Content Delivery Network}{https://galenguyer.com}
% \begin{itemize}
% \item Consists of Azure Virtual Machines distributed around the globe to provide the least latency based on the user's geographical location
% \item Deploys with Terraform for easily reproducible infrastructure
% \item Uses Ansible to distribute static files, configuration files, and SSL certificates from a central server to all edge servers
% \end{itemize}
% \textit{\textbf{Tools:} Azure Virtual Machines, Azure Traffic Managers, Bash, Nginx}

% \project{Kubernetes Cluster}{https://github.com/galenguyer/k8s}
% \begin{itemize}
% \item Deployed a Kubernetes cluster running on virtual machines managed by Proxmox with 3 control plane and 2 compute nodes
% \item Used HAProxy and PFSense for load balancing and routing
% \item Hosted a static website and a small web app written in Flask and backed by Redis for persistent data storage
% \end{itemize}
% \textit{\textbf{Tools:} Proxmox, Kubernetes, HAProxy, PFSense}

\clearpage

\end{document}
